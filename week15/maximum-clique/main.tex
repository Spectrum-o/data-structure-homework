%颜色没有大调整,跟校徽色不符
%只在参考的两个模板上改动了颜色,以及解决了参考文献的问题。
%[1]中所附的一些排版内容在本模板中删去,有需要可以从[1][2]中学习。
\documentclass{beamer}
\usepackage{ctex, hyperref}
\usepackage[T1]{fontenc}
    \documentclass{beamer}
\usepackage{tikz}
\usetikzlibrary{positioning, shapes.geometric, backgrounds, fit}

% 定义节点样式
\tikzset{
    basic/.style={circle, draw=black, thick, minimum size=7mm, inner sep=0pt, font=\bfseries},
    clique_node/.style={basic, fill=red!30, draw=red!80!black}, % 团/独立集节点(红)
    cover_node/.style={basic, fill=blue!30, draw=blue!80!black}, % 顶点覆盖节点(蓝)
    edge_g/.style={thick, black}, % 原图的边
    edge_gbar/.style={thick, black!60, dashed} % 补图的边
}
%\usepackage{cite}
% other packages
\usepackage{latexsym,amsmath,xcolor,multicol,booktabs,calligra}
\usepackage{biblatex}
\usepackage{graphicx,listings,stackengine}
\usefonttheme[onlymath]{serif}%统一字体
\addbibresource{ref.bib}
%可以多个bib文件
\author{钟以楠}
\title{maximum clique problems}
\institute{山东大学泰山学堂}
\date{2025年12月10日}
\usepackage{Tsinghua}
\usepackage[ruled,vlined]{algorithm2e}

% defs
\def\cmd#1{\texttt{\color{red}\footnotesize $\backslash$#1}}
\def\env#1{\texttt{\color{blue}\footnotesize #1}}
\definecolor{deepblue}{rgb}{0,0,0.5}
\definecolor{deepred}{rgb}{0.6,0,0}
\definecolor{deepgreen}{rgb}{0,0.5,0}
\definecolor{halfgray}{gray}{0.55}

\lstset{
    basicstyle=\ttfamily\small,
    keywordstyle=\bfseries\color{deepblue},
    emphstyle=\ttfamily\color{deepred},    % Custom highlighting style
    stringstyle=\color{deepgreen},
    numbers=left,
    numberstyle=\small\color{halfgray},
    rulesepcolor=\color{red!20!green!20!blue!20},
    frame=shadowbox,
}


\setbeamertemplate{bibliography item}[text]

\begin{document}

\kaishu
\begin{frame}
    \titlepage
    \begin{figure}[htpb]
        \begin{center}
            \includegraphics[width=0.3\linewidth]{logo2 .jpg}
        \end{center}
    \end{figure}
\end{frame}

\begin{frame}
    \tableofcontents[sectionstyle=show,subsectionstyle=show/shaded/hide,subsubsectionstyle=show/shaded/hide]
\end{frame}


\section{问题定义与理论基础}

% Slide 1: 背景与意义
\begin{frame}{背景与意义}
    \begin{itemize}
        \item \textbf{核心地位}:
            最大团问题 (Maximum Clique Problem, MCP) 是计算复杂性理论中最基础的 NP-hard 问题之一,也是 Karp (1972) 提出的 21 个经典 NPC 问题之一。
        \item \textbf{广泛应用}:
            \begin{itemize}
                \item \textbf{生物信息学}:蛋白质结构比对、基因序列分析。
                \item \textbf{社交网络}:社区发现、紧密群体识别。
                \item \textbf{编码理论}:纠错码设计、汉明距离约束。
            \end{itemize}
        \item \textbf{挑战性}:
            不仅精确求解困难,其近似求解也极具挑战性(Inapproximability)。
    \end{itemize}
\end{frame}

% Slide 2: 形式化定义
\begin{frame}{问题定义 (Formal Definition)}
    \begin{definition}[最大团 Maximum Clique]
        给定无向图 $G=(V, E)$,团 (Clique) 是顶点集 $V$ 的一个子集 $C \subseteq V$,满足:
        $$ \forall u, v \in C, u \neq v \implies \{u, v\} \in E $$
        即 $C$ 中的任意两个顶点都互为邻接点。
    \end{definition}
    
    \vspace{0.5cm}
    
    \begin{block}{关键区分}
        \begin{itemize}
            \item \textbf{极大团 (Maximal Clique)}:无法通过添加顶点扩展的团(局部极大值)。
            \item \textbf{最大团 (Maximum Clique)}:基数 $|C|$ 最大的团(全局最大值)。我们定义图 $G$ 的团数为 $\omega(G)$。
        \end{itemize}
    \end{block}
\end{frame}

\begin{frame}{核心概念:最大独立集 (MIS)}
    在讨论等价性之前,我们先定义补图中的对应概念:
    
    \vspace{0.5cm}

    \begin{block}{最大独立集 (Maximum Independent Set)}
        \begin{itemize}
            \setlength{\itemsep}{10pt} % 增加行间距,让排版更舒朗
            \item \textbf{定义}:
            给定无向图 $G=(V,E)$,独立集是顶点集 $S \subseteq V$,其中\textbf{任意两点均不相邻}。
            $$ \forall u, v \in S, \{u, v\} \notin E $$
            
            \item \textbf{直观理解}:
            一群“互不认识”的人,或者彼此之间没有连边的点。
            
            \item \textbf{记号}:
            其最大基数记为 $\alpha(G)$。
        \end{itemize}
    \end{block}
\end{frame}

% --- Slide 2: 最小顶点覆盖定义 ---
\begin{frame}{核心概念:最小顶点覆盖 (MVC)}
    这是理解等价性的难点,也是“互补”的关键:

    \vspace{0.5cm}

    \begin{alertblock}{最小顶点覆盖 (Minimum Vertex Cover)}
        \begin{itemize}
            \setlength{\itemsep}{10pt}
            \item \textbf{定义}:
            顶点集 $K \subseteq V$,使得图中的\textbf{每一条边}至少有一个端点属于 $K$。
            $$ \forall \{u, v\} \in E, \{u, v\} \cap K \neq \emptyset $$
            
            \item \textbf{直观隐喻 (Security Guards)}:
            想象在路口(顶点)设置安保人员,要求每一条街道(边)都必须处于至少一名安保人员的监控之下。
            
            \item \textbf{记号}:
            其最小基数记为 $\beta(G)$。
        \end{itemize}
    \end{alertblock}
\end{frame}

% --- Slide 3: 证明第一步 ---
\begin{frame}{等价性证明一:从团到独立集}
    \textbf{转化核心}:利用 \textbf{补图 (Complement Graph)} $\bar{G}$。
    
    \vspace{0.3cm}

    \begin{proof}
        设 $C \subseteq V$ 是图 $G$ 的一个顶点子集。
        \begin{itemize}
            \setlength{\itemsep}{5pt}
            \item $C$ 是 $G$ 的团 (全连接)
            \item $\iff \forall u, v \in C$, $u, v$ 在 $G$ 中\textbf{有}边相连
            \item $\iff \forall u, v \in C$, $u, v$ 在 $\bar{G}$ 中\textbf{无}边相连 (根据定义)
            \item $\iff C$ 是 $\bar{G}$ 的独立集
        \end{itemize}
    \end{proof}
    
    \begin{center}
        \textcolor{blue}{\textbf{结论 1: } $\omega(G) = \alpha(\bar{G})$}
    \end{center}
\end{frame}

% --- Slide 4: 证明第二步 ---
\begin{frame}{等价性证明二:从独立集到顶点覆盖}
    \textbf{转化核心}:利用 \textbf{集合的补 (Set Complement)} $V \setminus S$。


    \vspace{0.2cm}

    \begin{proof}
        \textbf{($\Rightarrow$)} 设 $S$ 为独立集(内部无边)。
        \begin{itemize}
            \item 图中每条边的两个端点,不可能\textbf{都}在 $S$ 中。
            \item $\implies$ 每条边至少有一个端点在 $S$ 的\textbf{外面} (即 $V \setminus S$)。
            \item $\implies V \setminus S$ 覆盖了所有的边。
        \end{itemize}
        
        \textbf{($\Leftarrow$)} 反之亦然。要使独立集 $|S|$ 最大,必须使覆盖集 $|V \setminus S|$ 最小。
    \end{proof}

    \begin{alertblock}{结论 (Final Equivalence)}
        $$ \omega(G) = \alpha(\bar{G}) = |V| - \beta(\bar{G}) $$
    \end{alertblock}
\end{frame}

\begin{frame}{最大团、最大独立集与最小顶点覆盖的关系}

\begin{center}
\resizebox{\textwidth}{!}{ % 自动缩放以适应屏幕
\begin{tikzpicture}

    %%% --- 左图:原图 G --- %%%
    \begin{scope}[local bounding box=GraphG]
        \node[above] at (1.5, 3.5) {\Large 原图 $G$: \textcolor{red}{最大团}};
        
        % 定义节点位置
        \node[clique_node] (A) at (90:1.5) {A};
        \node[clique_node] (B) at (210:1.5) {B};
        \node[clique_node] (C) at (330:1.5) {C};
        \node[basic] (D) at (2.5, 2) {D};
        \node[basic] (E) at (-2.5, 2) {E};

        % 绘制最大团的边 (完全连接)
        \draw[edge_g, ultra thick, red] (A) -- (B);
        \draw[edge_g, ultra thick, red] (B) -- (C);
        \draw[edge_g, ultra thick, red] (C) -- (A);

        % 绘制其他边 (噪音边)
        \draw[edge_g] (A) -- (D);
        \draw[edge_g] (B) -- (E);
        \draw[edge_g] (C) -- (D);
    \end{scope}

    %%% --- 右图:补图 G_bar --- %%%
    \begin{scope}[xshift=8cm, local bounding box=GraphGbar]
        \node[above] at (1.5, 3.5) {\Large 补图 $\bar{G}$: \textcolor{red}{MIS} 与 \textcolor{blue}{MVC}};

        % 相同的位置,但含义不同
        \node[clique_node] (A2) at (90:1.5) {A};
        \node[clique_node] (B2) at (210:1.5) {B};
        \node[clique_node] (C2) at (330:1.5) {C};
        \node[cover_node] (D2) at (2.5, 2) {D}; % 属于 MVC
        \node[cover_node] (E2) at (-2.5, 2) {E}; % 属于 MVC

        % 补图的边:原图没有的边,这里都有
        % 原图 A-B, B-C, C-A 有边 -> 这里无边 (独立集!)
        % 原图 A-E 无边 -> 这里有边
        \draw[edge_gbar] (A2) -- (E2);
        % 原图 B-D 无边 -> 这里有边
        \draw[edge_gbar] (B2) -- (D2);
        % 原图 C-E 无边 -> 这里有边
        \draw[edge_gbar] (C2) -- (E2);
        % 原图 D-E 无边 -> 这里有边
        \draw[edge_gbar] (D2) -- (E2);
        % 原图 A-D 有边 -> 这里无边
        % 原图 B-E 有边 -> 这里无边
        % 原图 C-D 有边 -> 这里无边
        
        % 所有的边至少有一个端点是蓝色的 (D或E) -> 这就是点覆盖
    \end{scope}

    % 分隔线
    \draw[dotted, thick, gray] (4, -2) -- (4, 4);

    % 说明文字
    \node[anchor=north, align=center, font=\small] at (1.5, -2) {
        \textbf{最大团} (Max Clique) \\
        $\{A, B, C\}$ 两两相连
    };

    \node[anchor=north, align=center, font=\small] at (9.5, -2) {
        \textbf{最大独立集} (MIS): \textcolor{red}{$\{A, B, C\}$} (互不相连) \\
        \textbf{最小点覆盖} (MVC): \textcolor{blue}{$\{D, E\}$} (覆盖所有虚线边) \\
        \textit{注: MVC = $V \setminus$ MIS}
    };

\end{tikzpicture}
}
\end{center}

\end{frame}

% Slide 4: 计算复杂性 - 展示理论深度
\begin{frame}{计算复杂性 (Computational Complexity)}
    \begin{itemize}
        \item \textbf{NP-Complete}: 
        判定是否存在大小为 $k$ 的团是 NPC 问题 (Karp, 1972)。
        
        \item \textbf{不可近似性 (Inapproximability)}:
        \begin{itemize}
            \item 除非 $P=NP$,否则不存在多项式时间近似方案 (PTAS)。
            \item Håstad (1999) 证明:对于任意 $\epsilon > 0$,在 $n^{1-\epsilon}$ 因子内近似 MCP 是 NP-hard 的。
        \end{itemize}
        
        \item \textbf{参数复杂性}:
        MCP 是 $W[1]$-complete 的,这意味着即使固定参数 $k$,也不太可能存在 $f(k) \cdot n^{O(1)}$ 的 FPT 算法。
    \end{itemize}
\end{frame}


\section{精确算法}
% --- Slide: 基于 DP 思想的算法 (Cliquer) ---

% --- Slide 1: 基础 B&B 框架 ---
% !!! 务必在导言区 (Preamble) 确保引入了以下宏包 !!!
% \usepackage[ruled, vlined, linesnumbered]{algorithm2e}
% \usepackage{xcolor}

% --- Slide 1.1: 基础框架 - 状态定义 ---
\begin{frame}{基础框架 :状态定义}
    几乎所有现代精确算法都基于通用的分支定界 (Branch-and-Bound) 框架。
    
    \vspace{0.3cm} % 稍微减小一点间距,防止内容太挤

    \begin{block}{核心递归状态 \texttt{Clique(C, P)}}
        算法维护两个关键集合和一个全局最优解:
        \begin{itemize}
            \setlength{\itemsep}{10pt} % 列表间距
            \item \textbf{当前团 $C$ (Current Clique)}:\\
            递归路径上已经确定选入团的顶点集合。
            
            \item \textbf{候选集 $P$ (Candidate Set)}:\\
            $P = \{v \in V \setminus C \mid \forall u \in C, (u,v) \in E\}$。\\
            即:剩余顶点中,与 $C$ 中\textbf{所有}顶点都相连的点集。
            
            \item \textbf{当前最优解 $C^*$ (Incumbent)}:\\
            搜索过程中发现的最大的团,用于剪枝比较。
        \end{itemize}
    \end{block}
\end{frame}

% --- Slide 1.2: 基础框架 - 算法流程 ---
\begin{frame}{基础框架 :Carraghan \& Pardalos (CP) 算法}
    % 定义全局变量关键字
    \SetKwInput{KwGlobal}{Global} 
    
    \begin{columns}[T] % T 表示顶部对齐
        % --- 左栏:文字描述 ---
        \begin{column}{0.4\textwidth}
            \vspace{0.5cm} % 稍微下沉一点
            \begin{itemize}
                \item \textbf{CP 算法 (1990)} 
                \item DFS 搜索的经典实现。
                \item 核心思想是利用\textcolor{red}{剪枝}策略减少搜索空间。
            \end{itemize}
        \end{column}
        
        % --- 右栏:算法代码 ---
        \begin{column}{0.6\textwidth}
            \begin{algorithm}[H]
                \footnotesize  % 整体缩小字号
                \SetAlgoLined
                \DontPrintSemicolon
                % 调整行距,如果还挤可以设为 0.9
                \renewcommand{\baselinestretch}{1.0} 
                
                \KwIn{当前团 $C$, 候选集 $P$}
                \KwGlobal{全局最优解 $C^*$}
                
                \BlankLine % 自动的空行,比 vspace 好
                
                \tcp{\textcolor{red}{1. 剪枝 (Bounding)}}
                \If{$|C| + |P| \le |C^*|$}{
                    \Return \tcp*[r]{\textcolor{gray}{无法更新最优解}} 
                }
                
                \tcp{\textcolor{blue}{2. 更新 (Update)}}
                \If{$|C| > |C^*|$}{
                    $C^* \leftarrow C$
                }
                
                \tcp{\textcolor{teal}{3. 分支 (Branching)}}
                \While{$P \neq \emptyset$}{
                    Pick $v \in P$, $P \leftarrow P \setminus \{v\}$\;
                    \texttt{Clique}($C \cup \{v\}, P \cap N(v)$)\;
                }
                
                \caption{CP Algorithm Structure}
            \end{algorithm}
        \end{column}
    \end{columns}
\end{frame}

% --- Slide 2: 剪枝进阶:着色上界 ---
% --- Slide 1: 理论与直觉 ---
\begin{frame}{核心剪枝 :基于着色的上界原理}
    
    % 使用 Block 强调核心定理
    \begin{alertblock}{Theorem: Weak Duality of Graphs}
        对于任意图 $G$,其最大团大小 $\omega(G)$ 总是小于等于其色数 $\chi(G)$:
        $$ \omega(G) \le \chi(G) $$
    \end{alertblock}

    \vspace{0.3cm}

    \begin{columns}[T]
        \begin{column}{0.48\textwidth}
            \textbf{\textcolor{blue}{直观理解 (Intuition)}}
            \begin{itemize}
                \item \textbf{团的性质}:团 $C$ 中的任意两个顶点都是相连的。
                \item \textbf{着色约束}:相连的顶点不能染相同颜色。
                \item \textbf{鸽巢原理}:如果你有一个大小为 $k$ 的团,你至少需要 $k$ 种不同的颜色来区分它们。
            \end{itemize}
        \end{column}
        
        \begin{column}{0.48\textwidth}
            \textbf{\textcolor{teal}{为什么有效?}}
            \begin{itemize}
                \item 相比于简单的 $|P|$ 上界,色数利用了图的\textbf{结构信息}。
                \item 颜色类 (Color Class) 实际上是一个\textbf{独立集} (Independent Set)。
                \item 一个团在每个独立集中至多只能选取 1 个顶点。
            \end{itemize}
        \end{column}
    \end{columns}

    \vspace{0.4cm}
    \centering
    \footnotesize{\textcolor{gray}{结论:如果我们能用 $k$ 种颜色给候选集 $P$ 合法着色,那么 $P$ 中最多只能选出 $k$ 个顶点加入团。}}
\end{frame}

% --- Slide 2: 工程实现与贪心策略 ---
\begin{frame}{核心剪枝 :贪心着色与工程实现}
    
    \textbf{挑战}:计算精确的 $\chi(G[P])$ 也是 NP-hard 问题。\\
    \textbf{对策}:使用\textbf{贪心着色 (Greedy Coloring)} 计算近似值。
    
    \vspace{0.2cm}

    \begin{columns}[T]
        \begin{column}{0.55\textwidth}
            \begin{algorithm}[H]
                \scriptsize
                \SetAlgoLined
                \DontPrintSemicolon
                \KwIn{候选集 $P$}
                \KwOut{颜色数 $k$}
                $k \leftarrow 0$\;
                \While{$P \neq \emptyset$}{
                    $k \leftarrow k + 1$\;
                    $I \leftarrow \text{MaxIndependentSet}(P)$ \tcp*[r]{贪心取}
                    \textcolor{gray}{// 将颜色 $k$ 分配给 $I$ 中的点}\;
                    $P \leftarrow P \setminus I$\;
                }
                \Return $k$
                \caption{Greedy Coloring Heuristic}
            \end{algorithm}
        \end{column}

        \begin{column}{0.4\textwidth}
            \vspace{0.2cm}
            \textbf{剪枝逻辑 (Pruning Logic)}:
            
            \vspace{0.2cm}
            Let $k_{greedy}$ be the result.
            $$ \omega(G[P]) \le \chi(G[P]) \le k_{greedy} $$
            
            \vspace{0.1cm}
            \begin{alertblock}{剪枝条件}
                \centering
                If $|C| + k_{greedy} \le |C^*|$ \\
                $\implies$ \textbf{Prune / Return}
            \end{alertblock}
            
            \vspace{0.1cm}
            \footnotesize{\textbf{复杂度}:通常为 $O(|P| + |E_P|)$,权衡了计算耗时与剪枝效果。}
        \end{column}
    \end{columns}
\end{frame}

% --- Slide 3: 里程碑算法 MCQ ---
\begin{frame}{ MCQ 算法详解 :着色与定界 (Numbering)}
    MCQ 的核心是一个名为 \texttt{NUMBER} 的过程。它不只是简单的排序,而是通过贪心着色计算每个点在当前子图中的\textbf{最大潜力}。
    
    \vspace{0.2cm}
    
    \begin{columns}[T]
        \begin{column}{0.55\textwidth}
            \begin{algorithm}[H]
                \scriptsize
                \SetAlgoLined
                \DontPrintSemicolon
                \KwIn{候选集 $P$}
                \KwOut{排序后的点集 $\{v_1, \dots, v_k\}$ 及各点色号}
                
                $i \leftarrow 1$, $max\_color \leftarrow 0$\;
                \While{$P \neq \emptyset$}{
                    Pick $v \in P$ (e.g., max degree in $P$)\;
                    \tcp{给 v 分配最小可用颜色}
                    $k \leftarrow \min \{c \ge 1 : c \notin \text{used\_colors}(N(v) \cap P)\}$\;
                    $color[v] \leftarrow k$\;
                    $P \leftarrow P \setminus \{v\}$, \textbf{push} $v$ to $Stack$\;
                    \If{$k > max\_color$}{ $max\_color \leftarrow k$ }
                }
                \caption{Procedure NUMBER(P)}
            \end{algorithm}
        \end{column}
        
        \begin{column}{0.42\textwidth}
            \small
            \textbf{过程解析}:
            \begin{itemize}
                \item \textbf{动态着色}:在每一层递归中,我们都对当前的 $P$ 重新着色。
                \item \textbf{颜色即上界}:
                这意味着包含 $v$ 的最大团,在当前剩余的 $P$ 中,最多还能再选 $color[v]-1$ 个点。
                \item \textbf{结果}:得到一个有序序列 $v_1, v_2, \dots$(栈的弹出顺序),通常颜色号大的点排在后面。
            \end{itemize}
        \end{column}
    \end{columns}
\end{frame}

\begin{frame}{ MCQ 算法详解 :递归与分支 (EXPAND)}
    % 将定义移到外面,节省 algorithm 环境内部空间
    \small \textbf{Input:} $P$ \quad \textbf{Global:} $C, C^*$
    
    % 极度压缩垂直间距
    \setlength{\textfloatsep}{0pt}
    \setlength{\intextsep}{0pt} 

    \begin{algorithm}[H]
        \scriptsize % 字体维持小号
        \renewcommand{\baselinestretch}{0.85} % 【关键】强制压缩行距
        \SetAlgoLined
        \DontPrintSemicolon
        
        \textbf{Procedure} \texttt{EXPAND}($P$) \;
        
        $\{v_1, \dots, v_k\} \leftarrow \texttt{NUMBER}(P)$ \;
        
        \BlankLine
        
        \For{$i \leftarrow k$ \textbf{downto} $1$}{
            $v \leftarrow v_i$ \;
            
            % 【关键】使用 \lIf 将判断和返回写在同一行
            \lIf{$|C| + color[v] \le |C^*|$}{ \Return } 
            
            \tcp{3. 递归与回溯}
            $P_{new} \leftarrow P \cap N(v)$ \;
            
            \If{$P_{new} = \emptyset$}{
                % 【关键】单行写法,省去3行空间
                \lIf{$|C| + 1 > |C^*|$}{ $C^* \leftarrow C \cup \{v\}$ }
            }
            \Else{
                \texttt{EXPAND}($P_{new}$) \;
            }
            
            $P \leftarrow P \setminus \{v\}$ \;
        }
    \end{algorithm}
    
    \vspace{0.1cm}
    \footnotesize
    \begin{itemize}
        \item \textbf{单行剪枝}:一旦 `color[v]` 降到阈值以下,循环直接结束(因为后续点颜色更小)。
        \item \textbf{基准情况}:当 $P_{new}$ 为空时,尝试更新全局最优解 $C^*$。
    \end{itemize}
\end{frame}

\begin{frame}{MCQ 算法详解 :原理证明与工程优化}
    \small % 整体缩小字号
    
    \begin{columns}[T] % T 表示顶部对齐
        % --- 左栏:数学原理 ---
        \begin{column}{0.48\textwidth}
            \textbf{\textcolor{blue}{1. 为什么 Color Bound 有效?}}
            \begin{itemize}
                \item \textbf{独立集性质}:着色本质上是将图划分为若干个独立集(颜色类)。
                \item \textbf{限制}:一个团在每个独立集中至多包含 1 个顶点。
                \item \textbf{结论}:如果 $P$ 能被 $k$ 种颜色着色,则 $\omega(P) \le k$。
            \end{itemize}

            \vspace{0.3cm} % 栏内可以稍微空一点,不影响整体

            \textbf{\textcolor{teal}{2. 逆序分支的数学意义}}
            \begin{itemize}
                \item 按 $color[v]$ 降序搜索 ($k \to 1$)。
                \item \textbf{快速收敛}:$color[v]$ 是当前分支的“硬上限”。随着循环进行,上限迅速下降。
                \item \textbf{早停}:一旦上限跌破 $|C^*| - |C|$,循环立即终止。
            \end{itemize}
        \end{column}
        
        % --- 右栏:工程实现 ---
        \begin{column}{0.48\textwidth}
            \textbf{\textcolor{red}{3. 工程瓶颈:集合求交}}
            \begin{itemize}
                \item 算法中最频繁的操作是计算新候选集:
                $$ P_{new} \leftarrow P \cap N(v) $$
                \item 在稠密图中,这是 $O(N)$ 的操作。
            \end{itemize}

            \vspace{0.2cm}
            
            \begin{alertblock}{关键优化:Bitset}
                \footnotesize
                利用 CPU 位并行特性加速集合运算。
                \begin{itemize}
                    \item \textbf{存储}:邻接矩阵用 \texttt{std::bitset} 或 \texttt{unsigned long long[]} 存储。
                    \item \textbf{运算}:交集变为按位与 (\texttt{\&})。
                    \item \textbf{效果}:复杂度从 $O(N)$ 降至 $O(N/64)$。
                \end{itemize}
                \centering
                \textit{这是现代最大团求解器 (如 MaxCliqueDyn) 速度提升百倍的基础。}
            \end{alertblock}
        \end{column}
    \end{columns}
\end{frame}

% --- Slide: 基于 DP 思想的算法 (Cliquer) ---
\begin{frame}{精确算法:基于 DP 思想的子问题求解}
    % 顶部引言保持全宽
    除了着色剪枝,另一类重要方法利用了\textbf{动态规划 (Dynamic Programming)} 的思想来复用计算结果。

    \vspace{0.3cm} % 稍微增加一点标题与内容的间距

    \begin{columns}[T] % T 表示顶部对齐
        % --- 左栏:定义与策略 ---
        \begin{column}{0.48\textwidth}
            \begin{block}{Cliquer 算法 (Östergård, 2002)}
                \textbf{核心策略}:反向归纳 (Iterative Deepening)
                \begin{enumerate}
                    \item 将顶点排序 $v_1, v_2, \dots, v_n$。
                    \item 定义后缀子图 $S_i = \{v_i, v_{i+1}, \dots, v_n\}$。
                    \item 定义状态 $c(i)$:\textbf{子图 $S_i$ 中的最大团大小}。
                \end{enumerate}
            \end{block}
        \end{column}

        % --- 右栏:伪代码 ---
        \begin{column}{0.5\textwidth}
            % 使用 minipage 有助于控制算法环境的边界,防止浮动问题
            \begin{algorithm}[H]
                \footnotesize % 缩小字体以适应半宽
                % \scriptsize % 如果依然超高,可以改用 \scriptsize
                \SetAlgoLined
                \DontPrintSemicolon % 可选:不打印分号,节省视觉杂乱感
                \KwData{顺序 $i = n \dots 1$}
                $c(n) = 1$ \;
                \For{$i = n-1$ \KwTo $1$}{
                    在 $S_i$ 找含 $v_i$ 最大团 $C_{new}$ \;
                    \tcp{利用 $c(i+1)$ 剪枝}
                    \If{$|Curr| + c(i+1) \le |Best|$}{
                        Prune (剪枝) \;
                    }
                    $c(i) = \max(|C_{new}|, c(i+1))$ \;
                }
                \caption{Östergård's Approach}
            \end{algorithm}
        \end{column}
    \end{columns}

    \vspace{0.3cm}

    % 底部注释保持全宽
    \centering
    \footnotesize{\textit{注:这虽非传统 $O(2^n)$ 状压 DP,但利用了已解子问题 $c(i+1)$ 来加速当前问题 $c(i)$ 的求解。}}
\end{frame}

% --- Slide 4: 现代优化技术 ---
\begin{frame}{现代优化:位并行与重着色}
    在 $N \approx 100 \sim 4000$ 的稠密图上,常数级优化至关重要。
    
    \vspace{0.3cm}

    \begin{itemize}
        \item \textbf{位并行 (Bit-parallelism) - BB-MaxClique}:
        \begin{itemize}
            \item \textbf{思路}:利用 CPU 的 64 位寄存器 (或 SIMD) 存储邻接矩阵。
            \item \textbf{操作}:集合求交集 $P \cap N(v)$ 转化为位运算 \texttt{P \& N[v]}。
            \item \textbf{效果}:相比传统数组实现,速度提升 10-30 倍,特别适合稠密图。
        \end{itemize}
        
        \vspace{0.2cm}

        \item \textbf{重着色 (Recoloring) - MCS}:
        \begin{itemize}
            \item \textbf{痛点}:贪心着色可能不准,导致界不够紧。
            \item \textbf{策略}:尝试将某些点的颜色“修复”到更小的颜色类中。如果能成功减少总色数,就能触发更多剪枝。
        \end{itemize}
    \end{itemize}
\end{frame}

\section{启发式算法与代数视角}

\begin{frame}{代数建模 :离散视角 (Integer Programming)}
    \footnotesize % 全局小字号
    \setlength{\abovedisplayskip}{3pt} % 压缩公式上方间距
    \setlength{\belowdisplayskip}{3pt} % 压缩公式下方间距
    \setlength{\textfloatsep}{0pt}

    最大团问题最自然的建模是基于补图的\textbf{独立集}约束 (Nemhauser \& Trotter, 1975)。

    % 直接展示公式,不用 Block 占用额外边距
    \begin{center}
        \fcolorbox{black}{gray!10}{
            \parbox{0.9\textwidth}{
                \centering
                Maximize \ $|C| = \sum x_i$ \quad s.t. \ $x_i + x_j \le 1, \ \forall \{i,j\} \notin E, \ x \in \{0,1\}^n$
            }
        }
    \end{center}

    \begin{columns}[T]
        % --- 左栏:几何含义 ---
        \begin{column}{0.48\textwidth}
            \textbf{\textcolor{blue}{1. 约束含义}}
            \begin{itemize}
                \setlength{\itemsep}{1pt} % 列表项紧贴
                \item $\{i,j\} \notin E \implies$ $i, j$ 不相连。
                \item 团要求两两相连,故不连通的对子中\textbf{至多选一个}。
                \item 这等价于求补图 $\bar{G}$ 的最大独立集。
            \end{itemize}
        \end{column}
        
        % --- 右栏:松弛缺陷 ---
        \begin{column}{0.48\textwidth}
            \textbf{\textcolor{red}{2. 线性松弛 (LP Relaxation)}}
            \begin{itemize}
                \setlength{\itemsep}{1pt}
                \item 若松弛为 $x_i \in [0,1]$,会出现巨大的 Integrality Gap。
                \item \textit{例}:$C_5$ (五边形),$\omega(G)=2$。
                \item LP 解可取全 $0.5$,目标值 $2.5$。
                \item \textbf{后果}:直接用 LP Solver 做 Branch \& Bound 效率极低。
            \end{itemize}
        \end{column}
    \end{columns}
\end{frame}

% --- Slide 4.1: 代数建模 ---
\begin{frame}{代数建模 :连续视角 (Motzkin-Straus)}
    \footnotesize
    \setlength{\abovedisplayskip}{2pt}
    \setlength{\belowdisplayskip}{2pt}

    1965 年,Motzkin-Straus 建立了组合优化与连续优化的桥梁。
    
    % 手动绘制高亮框,比 alertblock 省空间
    \vspace{0.1cm}
    \begin{center}
        \fcolorbox{red}{red!5}{
            \parbox{0.95\textwidth}{
                \textbf{Theorem:} 在标准单纯形 $\Delta = \{x \in \mathbb{R}^n \mid \sum x_i = 1, x_i \ge 0\}$ 上最大化 $f(x) = x^T A x$。若最大值为 $f^*$,则:
                $$ \omega(G) = \frac{1}{1 - f^*} \iff f^* = 1 - \frac{1}{\omega(G)} $$
            }
        }
    \end{center}
    \vspace{0.1cm}

    \begin{columns}[T]
        \begin{column}{0.5\textwidth}
            \textbf{\textcolor{teal}{直观推导 (Intuition)}}
            \begin{itemize}
                \setlength{\itemsep}{1pt} 
                \item 假设团 $C$ 大小为 $k$,我们在 $C$ 上\textbf{均匀分配}概率。
                \item 令 $x_i = 1/k$ (若 $i \in C$),否则 $0$。
                \item 代入 $f(x) = \sum A_{ij} x_i x_j$。
            \end{itemize}
        \end{column}
        
        \begin{column}{0.48\textwidth}
            \textbf{\textcolor{teal}{计算过程}}
            \begin{itemize}
                \setlength{\itemsep}{1pt}
                \item 团内有 $k(k-1)$ 条有向边($A_{ij}=1$)。
                \item 每一项贡献 $(1/k) \cdot (1/k) = 1/k^2$。
                \item 总和:$k(k-1) \times \frac{1}{k^2} = 1 - \frac{1}{k}$。
            \end{itemize}
        \end{column}
    \end{columns}
    
    \vspace{0.2cm}
    \centering
    \scriptsize \fcolorbox{gray}{white}{结论:团越大 ($k \uparrow$) $\implies$ 函数值越高 ($f \to 1$)}
\end{frame}


% --- Slide 4.2: 求解困境 ---
\begin{frame}{数学困境 :非凸性的本质 (Non-Convexity)}
    \footnotesize % 1. 缩小字号
    试图最大化 $f(x) = x^T A x$ ($x \in \Delta$) 时,我们将面临极不友好的几何性质。

    % 2. 使用 alertblock 强调数学性质
    \begin{alertblock}{1. 谱性质:不定矩阵 (Indefinite Matrix)}
        对于任意非平凡图,邻接矩阵 $A$ 既非正定也非半正定。
        \begin{itemize}
            \setlength{\itemsep}{1pt} % 让列表更紧凑
            \item 这意味着特征值 $\lambda$ \textbf{有正有负}。
            \item 导致优化曲面呈现复杂的“马鞍面”特征,而非碗状。
        \end{itemize}
    \end{alertblock}

    % 3. 使用 block 解释后果
    \begin{block}{2. 几何后果:崎岖的优化地形}
        由于矩阵不定,目标函数 $f(x)$ \textbf{非凸非凹},导致:
        \begin{itemize}
            \setlength{\itemsep}{2pt}
            \item \textbf{局部陷阱}:空间中充满指数级数量的\textbf{局部极大值}。
            \item \textbf{致命的对应关系}:
            $$ \text{局部极值} \iff \text{\textcolor{blue}{极大团 (Maximal)}} $$
            $$ \text{全局极值} \iff \text{\textcolor{red}{最大团 (Maximum)}} $$
            \item \textbf{结论}:简单的梯度上升只能带你找到“极大团”,几乎不可能翻越山岭找到“最大团”。
        \end{itemize}
    \end{block}
\end{frame}


\begin{frame}{数学困境 :微积分与线性代数的失效}
    \small
    既然函数性质不好,标准数学工具的表现如何?

    \vspace{0.2cm}

    \begin{columns}[T]
        % --- 左栏:硬核数学分析 ---
        \begin{column}{0.55\textwidth}
            \textbf{1. 梯度上升 (Gradient Ascent)}
            \begin{itemize}
                \item 梯度 $\nabla f(x) = 2Ax$ 指向当前增长最快的方向。
                \item \textbf{缺陷}:对于多峰函数,梯度上升是“短视”的。它会迅速收敛到最近的极大团(Local Optima),然后停滞不前。
            \end{itemize}
            
            \vspace{0.2cm}
            
            \textbf{2. 谱分析 (Spectral Analysis)}
            \begin{itemize}
                \item \textbf{Wilf Bound}: $\omega(G) \le \lambda_{max} + 1$。
                \item \textbf{缺陷}:这只是一个\textbf{上界}。特征向量对应的是实数域的最优解,很难直接映射回离散的 0/1 团结构。
            \end{itemize}
        \end{column}
        
        % --- 右栏:直观比喻 ---
        \begin{column}{0.4\textwidth}
            \vspace{0.3cm}
            \begin{beamerboxesrounded}[shadow=true]{直观理解:登山悖论}
                \centering \footnotesize
                想象一片连绵起伏的山脉:
                \vspace{0.1cm}
                \begin{itemize}
                    \item \textbf{极大团} = 无数个小山头。
                    \item \textbf{最大团} = 珠穆朗玛峰。
                \end{itemize}
                \vspace{0.1cm}
                微积分(梯度)就像\textbf{蒙着眼的登山者},只能感知脚下的坡度。他大概率会爬上家门口的小土坡,然后以为自己到了最高点。
            \end{beamerboxesrounded}
            
            \vspace{0.3cm}
            \centering
            \fcolorbox{red}{white}{\small $\therefore$ 必须引入\textbf{扰动}与\textbf{跳跃} (启发式)}
        \end{column}
    \end{columns}
\end{frame}


% --- Slide 4.3: 启发式策略分类 ---
\begin{frame}{现代启发式框架 :罚函数策略 (k-fixed Penalty)}
    \footnotesize
    \setlength{\textfloatsep}{0pt}
    
    这类算法(如 NuMVC, FastStep)将优化问题转化为一系列\textbf{判定问题}:能否找到大小为 $k$ 的团?
    
    \vspace{0.1cm}

    \begin{columns}[T]
        \begin{column}{0.48\textwidth}
            \begin{block}{1. 核心机制:允许“犯错”}
                \begin{itemize}
                    \setlength{\itemsep}{1pt}
                    \item \textbf{状态定义}:
                    搜索空间包含非团子图。我们在 $V$ 中任选 $k$ 个点作为集合 $S$。
                    \item \textbf{能量函数 (Cost Function)}:
                    $$ f(S) = |\{(u,v) \in S \times S \mid (u,v) \notin E\}| $$
                    即 $S$ 中缺失边的数量(Missing Edges)。
                    \item \textbf{目标}:通过局部调整使得 $f(S) = 0$。
                \end{itemize}
            \end{block}
        \end{column}
        
        \begin{column}{0.48\textwidth}
            \begin{alertblock}{2. 深度直觉:填海造陆}
                \begin{itemize}
                    \setlength{\itemsep}{1pt}
                    \item \textbf{困境}:合法团之间往往被“非团结构”隔开(像海洋隔开群岛)。如果只允许走合法团,很容易卡死。
                    \item \textbf{策略}:罚函数允许算法暂时进入“非法区域”(海里),以此为桥梁走到另一个更好的团。
                    \item \textbf{Configuration Checking}:为了防止在海里迷路(循环震荡),必须引入禁忌机制 (Tabu)。
                \end{itemize}
            \end{alertblock}
        \end{column}
    \end{columns}
\end{frame}

\begin{frame}{现代启发式框架 :合法策略 (Legal Strategy)}
    \footnotesize
    这一流派始终维护当前解 $C$ 为\textbf{合法团}。其算子设计深受连续优化梯度 ($\nabla f = 2Ax$) 的启发。

    \vspace{0.1cm}

    \begin{block}{三大核心算子 (Operators)}
        \begin{description}
            \setlength{\itemsep}{3pt}
            \item[\textbf{1. Add (扩展)}] 
            若存在点 $v$ 与 $C$ 全连通,直接加入。
            \textit{(对应贪心上升)}
            
            \item[\textbf{2. Swap (交换) - 梯度的离散化身}] 
            当无法 Add 时(陷入极大团),我们需要换点。
            \begin{itemize} \scriptsize
                \item \textbf{如何选点?} 计算候选集中每个点 $v$ 的权重(通常基于 $x^T A x$ 的梯度增益)。
                \item \textbf{操作}:踢出一个“最差”的点,换入一个“潜力最大”的点。
                \item \textbf{意义}:这是在离散空间中模拟沿着梯度方向修正搜索路径。
            \end{itemize}

            \item[\textbf{3. Drop (扰动)}] 
            如果 Swap 多次后仍无进展,强制随机丢弃 $C$ 中部分节点。
            \textit{(对应跳出局部最优 Basin)}
        \end{description}
    \end{block}
    
    \centering
    \scriptsize \textcolor{blue}{总结:合法策略更像是一个“带随机扰动的爬山算法”。}
\end{frame}

\begin{frame}{现代启发式框架 :进化算法的陷阱与出路}
    \footnotesize
    
    在最大团问题 (MCP) 上,纯粹的遗传算法 (GA) 往往不如简单的局部搜索。
    
    \vspace{0.1cm}

    \begin{columns}[T]
        \begin{column}{0.5\textwidth}
            \textbf{\textcolor{red}{1. 结构脆弱性与交叉失效}}
            \begin{itemize}
                \setlength{\itemsep}{2pt}
                \item \textbf{核心矛盾}:团是一个\textbf{高内聚}结构。
                \item \textbf{交叉算子 (Crossover)}:
                假设父代 $P_1$ 和 $P_2$ 都是团。
                \item \textbf{语义破坏}:
                简单的杂交(如各取一半)产生的子代,往往内部极其稀疏,支离破碎。
                \item \textbf{后果}:
                子代不再是团,修复它需要耗费巨大计算量。算法退化为随机搜索。
            \end{itemize}
        \end{column}
        
        \begin{column}{0.45\textwidth}
            \textbf{\textcolor{teal}{2. 破局:模因算法 (Memetic)}}
            \begin{itemize}
                \setlength{\itemsep}{2pt}
                \item 既然 GA 搜得慢,局部搜索 (LS) 容易卡住,不如结合。
                \item \textbf{框架}:
                $$ \text{GA (负责跳跃)} + \text{LS (负责挖掘)} $$
                \item \textbf{专用重组算子}:
                现代算法 (如 Map-Elites) 设计了保留团结构的特殊交叉算子,而非简单的基因片段交换。
            \end{itemize}
        \end{column}
    \end{columns}

    \vspace{0.3cm}
    \centering
    \fcolorbox{red}{white}{
        \parbox{0.9\textwidth}{
            \centering \scriptsize \textbf{Key Insight}: 
            不要直接套用通用 AI 算法。算法设计必须尊重问题的\textbf{拓扑结构 (Topology)}。
        }
    }
\end{frame}

\section{现实应用}

\begin{frame}{现实应用 :网络分析与生物计算}
    \small
    \setlength{\textfloatsep}{0pt}
    
    \begin{columns}[T] % T 表示顶部对齐,互不干扰
        % --- 左栏:社交网络 ---
        \begin{column}{0.48\textwidth}
            \begin{block}{1. 社交网络 (SNA)}
                \textbf{场景}:在海量数据中挖掘核心圈层。
                \begin{itemize}
                    \setlength{\itemsep}{2pt}
                    \item \textbf{社区发现}:
                    寻找高密度的\textbf{Cohesive Subgroups}。
                    \item \textbf{反恐与风控}:
                    恐怖组织或黑产团伙通常表现为全连通子图(团)。
                    \item \textbf{难点}:
                    图规模极大(百万节点),需结合启发式算法(如 k-core 预处理)进行剪枝。
                \end{itemize}
            \end{block}
        \end{column}
        
        % --- 右栏:生物信息 ---
        \begin{column}{0.48\textwidth}
            \begin{block}{2. 生物信息 (Bioinformatics)}
                \textbf{场景}:分子结构的相似性比对。
                \begin{itemize}
                    \setlength{\itemsep}{2pt}
                    \item \textbf{蛋白质结构预测}:
                    判断两个蛋白质在空间上是否同构。
                    \item \textbf{建模手段}:
                    构建\textbf{对应图 (Correspondence Graph)}。
                    \begin{itemize} \scriptsize
                        \item 节点:原子对 $(u, v)$。
                        \item 边:若距离约束兼容则连线。
                    \end{itemize}
                    \item \textbf{目标}:求最大团即求最大公共子结构 (MCS)。
                \end{itemize}
            \end{block}
        \end{column}
    \end{columns}
\end{frame}
\begin{frame}{现实应用 :计算机视觉与信息论}
    % 1. 使用 footnote size (比 small 更小一点,非常适合双栏内容)
    \footnotesize 
    
    \begin{columns}[T]
        % --- 左栏:计算机视觉 ---
        \begin{column}{0.48\textwidth}
            \begin{block}{3. 计算机视觉 (CV)}
                \textbf{场景}:图像匹配与物体识别。
                \begin{itemize}
                    % 2. 压缩列表项之间的垂直距离
                    \setlength{\itemsep}{1pt} 
                    \item \textbf{特征匹配}:
                    寻找两幅图中几何一致的特征点集。
                    \item \textbf{关联图 (Association Graph)}:
                    \begin{itemize} \scriptsize
                        \item 节点:假设匹配对 $(u, v)$。
                        \item 边:若几何约束兼容则连线。
                    \end{itemize}
                    \item \textbf{本质}:在含噪图中寻找最大相容子集。
                \end{itemize}
            \end{block}
        \end{column}
        
        % --- 右栏:编码理论 ---
        \begin{column}{0.48\textwidth}
            \begin{block}{4. 编码理论 (Coding)}
                \textbf{场景}:零误差信道传输。
                \begin{itemize}
                    \setlength{\itemsep}{1pt}
                    \item \textbf{纠错设计}:
                    选一组信号,使任意两个都不会因干扰混淆。
                    \item \textbf{香农容量 (Shannon Capacity)}:
                    \begin{itemize} \scriptsize
                        \item 建“混淆图” $G$ (若混淆则连边)。
                        \item 问题转化为求 $\alpha(G)$ (最大独立集)。
                    \end{itemize}
                    \item \textbf{理论}:Lovász $\vartheta$ 函数为此提供紧上界。
                \end{itemize}
            \end{block}
        \end{column}
    \end{columns}

    % 3. 减少垂直间距 (从 0.3cm 改为 0.1cm)
    \vspace{0.1cm}
    
    % --- 底部总结 ---
    \centering
    \fcolorbox{blue}{white}{
        \parbox{0.95\textwidth}{
            % 4. 总结框内文字稍微小一点,确保不换行太多
            \centering \scriptsize 
            \textbf{Insight}:现实中的“兼容性 (Compatibility)”与“互不冲突”,
            在数学上往往等价于图论中的\textbf{最大团}或\textbf{独立集}问题。
        }
    }
\end{frame}

\section{总结}

\begin{frame}{全景回顾:精度与速度的权衡}
    \footnotesize
    最大团问题的求解历史,本质上是在\textbf{搜索空间剪枝}与\textbf{局部极值跳出}之间寻找平衡。

    \vspace{0.1cm}

    \begin{columns}[T]
        % --- 左栏:精确算法 ---
        \begin{column}{0.48\textwidth}
            \begin{block}{1. 精确算法 (Exact: B\&B)}
                \textbf{代表}:MCQ, MaxCliqueDyn
                \begin{itemize}
                    \item \textbf{核心}:利用数学界(着色、MaxSAT)进行剪枝。
                    \item \textbf{优势}:保证最优性 ($C^*$)。
                    \item \textbf{劣势}:最坏复杂度 $O(3^{n/3})$,难以处理 $N > 4000$ 的稠密图。
                \end{itemize}
            \end{block}
        \end{column}
        
        % --- 右栏:启发式算法 ---
        \begin{column}{0.48\textwidth}
            \begin{block}{2. 启发式 (Heuristic: LS)}
                \textbf{代表}:NuMVC, FastStep
                \begin{itemize}
                    \item \textbf{核心}:利用扰动机制(Penalty, Swap)跳出局部最优。
                    \item \textbf{优势}:能在巨大图上迅速找到高质量解。
                    \item \textbf{劣势}:无最优性证明。
                \end{itemize}
            \end{block}
        \end{column}
    \end{columns}

    \vspace{0.2cm}

    \begin{alertblock}{3. 现代趋势:混合求解 (Hybrid Solving)}
        现在的顶级求解器倾向于结合两者:
        \begin{itemize}
            \item 使用\textbf{启发式}快速找到一个大的 LB (Lower Bound)。
            \item 将此 LB 注入\textbf{精确算法}作为初始界,大幅提升剪枝效率。
        \end{itemize}
    \end{alertblock}
\end{frame}

\begin{frame}{前沿展望:当组合优化遇上 AI}
    \small
    随着图神经网络 (GNN) 和深度强化学习 (DRL) 的兴起,MCP 研究正在进入新的范式。

    \setlength{\textfloatsep}{0pt} 

    \begin{columns}[T]
        \begin{column}{0.6\textwidth}
            \textbf{1. Learning to Branch / Prune}
            \begin{itemize}
                \item 传统 MCQ 使用贪心着色来排序分支。
                \item \textbf{AI 思路}:训练 GNN 预测哪个节点最可能在最大团中,从而指导分支顺序。
                \item \textit{优势:在特定分布的图上,比人工设计的 Heuristic 快数个数量级。}
            \end{itemize}
            
            \vspace{0.2cm}
            
            \textbf{2. 连续优化的进阶 (SDP)}
            \begin{itemize}
                \item 除了 Motzkin-Straus,\textbf{半正定规划 (SDP)} 提供了更紧的界(Lovász Theta function $\vartheta(G)$)。
                \item 这是一个凸松弛 (Convex Relaxation),可以在多项式时间内求解。
            \end{itemize}
        \end{column}
        
        \begin{column}{0.35\textwidth}
            \vspace{0.2cm}
            \begin{beamerboxesrounded}[shadow=true]{Open Problems}
                \centering \scriptsize
                \begin{enumerate}
                    \item \textbf{大规模稀疏图}:社交网络图的团挖掘 (Web-scale)。
                    \item \textbf{端到端求解}:能否完全用神经网络替代搜索?(目前还很难)。
                    \item \textbf{硬件加速}:利用 GPU/FPGA 并行化 Bitset 运算。
                \end{enumerate}
            \end{beamerboxesrounded}
        \end{column}
    \end{columns}
    
    \vspace{0.3cm}
    \centering
    \footnotesize \textit{The journey from classical algorithms to AI-driven optimization continues.}
\end{frame}


\begin{frame}
    \begin{center}
        {\Huge\calligra Thanks!}
    \end{center}
\end{frame}

\end{document}